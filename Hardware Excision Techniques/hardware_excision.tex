% iaus2esa.tex -- sample pages for Proceedings IAU Symposium document class
% (based on v1.0 cca2esam.tex)
% v1.04 released 17 May 2004 by TechBooks
%% small changes and additions made by KAvdH/IAU 4 June 2004
% Copyright (2004) International Astronomical Union

\NeedsTeXFormat{LaTeX2e}

\documentclass{iau_FM}
\usepackage{amsmath}
\usepackage{graphicx}
\usepackage{multirow}
\usepackage{natbib}
\usepackage{journals}

\bibliographystyle{aasjournal}

\title[Hardware excision] %% give here short title %%
{RFI hardware excision techniques and their limitations}

\author[Bla X. Blub et al.]   %% give here short author list %%
{Jane D. Author1$^1$
%%  \thanks{Present address:},
 \and Jim D. Author2$^2$}

\affiliation{
$^1$ Affiliation1 Name \\ Affiliation2 Address \\ email: {\tt bla2@blub.bla}\\[\affilskip]
$^2$ Affiliation2 Name \\ Affiliation2 Address \\ email: {\tt bla2@blub.bla}\\[\affilskip]
}

\pubyear{2024}
\setcounter{page}{1}
\jname{Somejournal} 
\editors{Gyula I. G. J\'ozsa, Michael Lindqvist, Anastasios Tzioumis, eds.}
\begin{document}

\maketitle

\begin{abstract}
Abstract
\keywords{Keyword1, keyword2, keyword3, etc.}
%% add here a maximum of 10 keywords, to be taken form the file <Keywords.txt>
\end{abstract}

\firstsection % if your document starts with a section,
              % remove some space above using this command.
\section{Introduction}
Start text here, example citation: \citep{murthy21}
\section{RFI mitigation in real-time}
Which frequencies are covered most in radio astronomy and which type of telescopes receivers are used and will be developed?
\section{Limitations}
\begin{table}
  \begin{center}
  \caption{Overview of current knowledge on circum-stellar condensate grains in meteorites.}
  \label{tab1}
 {\scriptsize
  \begin{tabular}{|l|c|c|c|c|}\hline 
{\bf Mineral} & {\bf Size [$\mu$m]} & {\bf Isotopic Signatures} & {\bf Stellar} & {\bf Contri-} \\ 
   &  {\bf abund.}  [ppm]$^1$ & & {\bf Sources} & {\bf bution$^2$} \\ \hline
diamond & $~0.0026$ & Kr-H, Xe-HL, Te-H & supernovae & ? \\
   & ~1500 & & & \\ \hline
silicon & $~0.1-10$ & enhanced $^{13}$C, $^{14}$N, $^{22}$Ne, s-process elem. & AGB stars & $> 90$~\% \\
carbide & $~30$ & low $^{12}$C/$^{13}$C, often enh.\ $^{15}$N & J-type C-stars (?) & $< 5$~\% \\
 & & enhanced $^{12}$C, $^{15}$N, $^{28}$Si; extinct $^{26}$Al, $^{44}$Ti & Supernovae & 1~\% \\
 & & low $^{12}$C/$^{13}$C, low $^{14}$N/$^{15}$N & novae &  $0.1$~\% \\ \hline
graphite & $~0.1-10$ & enh.\ $^{12}$C, $^{15}$N, $^{28}$Si; extinct $^{26}$Al, $^{41}$Ca, $^{44}$Ti &
 SN (WR?) & $< 80$~\% \\ 
 & $~10$ & s-process elements & AGB stars & $> 10$~\% \\
 & & low $^{12}$C/$^{13}$C  & J-type C-stars (?) & $< 10$~\% \\
 & & low $^{12}$C/$^{13}$C; Ne-E(L) & novae & 2~\% \\ \hline
 corundum/ & $~0.1-5$ & enhanced $^{17}$O, moderately depl. $^{18}$O & RGB / AGB & $> 70$~\% \\
 spinel/ & $~50$ & enhanced $^{17}$O, strongly depl. $^{18}$O & AGB stars & 20~\% \\
 hibonite & & enhanced $^{16}$O & supernovae & 1~\% \\ \hline
 silicates & $~0.1-1$ & similar to oxides above & & \\
 &  $~140$ & & & \\ \hline
 silicon & $~1$ & enhanced $^{12}$C, $^{15}$N, $^{28}$Si; extinct $^{26}$Al & supernovae & 100~\% \\
 nitride & $~ 0.002$ & & & \\ \hline
  \end{tabular}
  }
 \end{center}
\vspace{1mm}
 \scriptsize{
 {\it Notes:}\\
  $^1$For the abund.\ (in wt.\ ppm) the reported maximum values from different meteorites are given. \\
  $^2$Note uncertainty about actual fraction of diamonds that are pre-solar and for fraction of graphite attributed to SN and AGB stars (see discussion in text).}
\end{table}

\bibliography{hardware_excision}

%\begin{thebibliography}{}
%
%\bibitem[Amari \etal\ (1995)]{Amari_etal95}
%{Amari, S., Hoppe, P., Zinner, E., \& Lewis R.S.} 1995,
%\textit{Meteoritics}, 30, 490 

%\end{thebibliography}

%\begin{discussion}
%\discuss{someone}{bla}
%\end{discussion}

\end{document}
