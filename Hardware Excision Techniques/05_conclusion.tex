\section{Conclusion}
\label{section:hardware:conclusion}

The future of radio astronomy faces significant challenges as the radio spectrum becomes increasingly congested with anthropogenic signals. The evolving landscape of RFI demands that observatories implement a variety of mitigation strategies, from real-time excision techniques to preventive measures like dynamic scheduling and collaborative frameworks with spectrum users.
The chapter provided an overview of the effects of RFI on radio telescope receivers, the classification of RFI, the need for real-time mitigation, and the contemporary real-time RFI mitigation techniques. As part of the real-time implementation, we covered analog and digital techniques in the radio telescope receiver systems. The chapter provides details on the contemporary techniques commissioned and proposed in various telescopes across the world. Many techniques, particularly for mitigating strong RFI in the analog domain, have yielded good results. Techniques in downstream digital signal processing have also shown benefits, particularly the Excision class. These are mostly the ones commissioned and used at the radio telescopes. More advanced signal processing techniques and recent ones using AI and ML are being explored. A summary table is presented at the end of the chapter.

Looking forward, integrating adaptive RFI mitigation into telescope design, alongside preventive measures like satellite boresight avoidance and reconfigurable intelligent surfaces, will be essential. As we push the boundaries of sensitivity in radio astronomy, robust RFI management will contribute to the protection of the integrity of observations and ensure the sustainability of astronomical research in an increasingly crowded spectrum.

%The future of radio astronomy faces significant challenges as the radio spectrum becomes increasingly congested with anthropogenic signals. The evolving landscape of RFI demands that observatories implement a variety of mitigation strategies, from real-time excision techniques to preventive measures like dynamic scheduling and collaborative frameworks with spectrum users. Existing techniques, such as adaptive analog attenuators, pre-LNA superconducting filters, and switchable notch filters, have proven effective in attenuating interference, but each comes with trade-offs in sensitivity and operational complexity. Furthermore, digital RFI mitigation strategies, such as spectral kurtosis, time-domain parametric subtraction, and UV-domain flagging, have expanded our ability to detect and excise RFI with minimal loss of astronomical data.

%Ongoing efforts in real-time flagging, cyclic spectroscopy, or spectral kurtosis, exemplify advanced mitigation techniques tailored to the unique RFI challenges encountered in high-precision pulsar timing and studies of the interstellar medium. Similarly, the use of tunable notch filters, spatial filtering, and artificial intelligence-based pattern recognition holds promise for next-generation arrays like DSA-2000 and SKA-Mid, where complex RFI environments require adaptive, multi-faceted approaches.

%Looking forward, integrating adaptive RFI mitigation into telescope design, alongside preventive measures like satellite boresight avoidance and reconfigurable intelligent surfaces, will be essential. As we push the boundaries of sensitivity in radio astronomy, robust RFI management will not only protect the integrity of observations but also ensure the sustainability of astronomical research in an increasingly crowded spectrum.