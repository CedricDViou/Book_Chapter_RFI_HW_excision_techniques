\section{Summary table} %(Kaushal, All to contribute)
\label{section:hardware:summary}


\begin{longtable}{|p{3cm}|p{4cm}|p{5cm}|p{5cm}|}
\hline
\textbf{Category} & \textbf{Technique} & \textbf{Description} & \textbf{Applications/Advantages} \\
\hline
\endfirsthead
\hline
\textbf{Category} & \textbf{Technique} & \textbf{Description} & \textbf{Applications/Advantages} \\
\hline
\endhead
\hline
\endfoot

\textbf{Avoidance} & Automated Scheduling & Avoids known fixed/moving RFI sources by aligning telescope observations with interference models. & Optimizes observation schedules to minimize RFI impact, particularly useful for fixed sources and dynamic RFI like LEO satellites. \\
\hline
\textbf{Avoidance} & Dynamic Scheduling & Adapts schedules in real time based on RFI conditions. & Used by GBT and MeerKAT to prioritize cleaner spectral windows and mitigate dynamic RFI. \\
\hline
\textbf{Avoidance} & Satellite Boresight Avoidance & Adjusts telescope pointing and satellite emission control to reduce interference. & Tested at GBT with Starlink satellites using real-time data sharing via the Operational Data Sharing (ODS) system. \\
\hline
\textbf{Analog} & Adaptive Analog Attenuators & Dynamically adjusts attenuation to prevent receiver saturation. & Used by OVRO-LWA to mitigate strong daytime RFI, preserving baseline levels while limiting sensitivity loss. \\
\hline
\textbf{Analog} & Pre-LNA Superconducting Filters & High-temperature superconducting filters before LNA for selective RFI suppression. & Applied at Yebes Observatory to minimize RFI, though limited by spurious responses and lack of automation. \\
\hline
\textbf{Analog} & Front-End Notch Filters & Rejects narrowband RFI before amplification. & Used in GMRT to suppress strong RFI, ensuring linear receiver operation. \\
\hline
\textbf{Analog} & Switchable Notch Filters & Attenuates specific RFI bands dynamically. & GBT employs these filters to suppress C-band interference and provides real-time RFI monitoring tools. \\
\hline
\textbf{Analog} & Tunable Notch Filters & Voltage-controlled filters for precise RFI mitigation. & Designed for DSA-2000 to cover 550-1050 MHz with up to 62.5 dB attenuation. \\
\hline
\textbf{Analog} & Reconfigurable Intelligent Surfaces (RIS) & Dynamically cancels RFI through wavefront shaping. & Creates electromagnetic quiet zones without affecting astronomical signals. \\
\hline
\textbf{Digital} & Real-Time RFI Excision & Removes impulsive RFI in time-domain data. & Used by uGMRT with FPGA-based systems for broadband RFI mitigation, improving sensitivity by filtering out impulses in real time. \\
\hline
\textbf{Digital} & Post-Channelization Flagger & Identifies and flags non-linearities and spectral anomalies. & Implemented in DSA-2000 to ensure accurate visibility data, improving calibration and imaging fidelity. \\
\hline
\textbf{Digital} & Sigma Cut Flagging & Detects RFI using robust statistical thresholds in time and frequency domains. & RFIm and Apertif Radio Transient System (ARTS) use this approach for high-throughput, effective RFI detection. \\
\hline
\textbf{Digital} & Spectral Kurtosis (SK) & Detects non-Gaussian signals for real-time RFI identification. & Applied at EOVSA for dynamic RFI flagging while preserving astronomical data. \\
\hline
\textbf{Digital} & MeFisTo Filtering & Median filtering in time and frequency domains to mitigate narrowband and wideband RFI. & Used at Nançay Decameter Array for solar and Jovian observations in low-frequency bands. \\
\hline
\textbf{Digital} & Power-Based Excision & Flags strong RFI bursts and integrates clean data. & Developed at Nançay for HI redshift surveys and OH megamasers amidst radar and satellite RFI. \\
\hline
\textbf{Digital} & Calibration VarThreshold & Advanced 2D flagging algorithm for time and frequency. & Used by MeerKAT to ensure sensitivity by rejecting 20\% of contaminated L-band data during calibration. \\
\hline
\textbf{Digital} & GRIDflag (UV-Domain Flagging) & Flags RFI directly in the UV domain based on statistical comparisons. & Enhances sensitivity and imaging fidelity for interferometric arrays. \\
\hline
\textbf{Digital} & Spatial Filtering & Suppresses RFI using spatial diversity and unique signal subspaces. & Effective for dense arrays like LOFAR, mitigating human-made RFI with minimal impact on astronomical signals. \\
\hline
\textbf{Digital} & Collaborative Signal Subtraction & Removes RFI by sharing interference models with sources like cellular networks. & Reduces computational overhead while recovering clean astronomical data, effective in environments with multiple RFI sources. \\
\hline
\textbf{Digital} & Pattern Recognition & Identifies RFI through edge detection and statistical analysis. & Real-time implementation for dynamic flagging, suitable for highly contaminated environments. \\
\hline
\textbf{Digital} & Real-Time Satellite Prediction & Predicts and flags satellite passes over the telescope beam. & Implemented at GMRT to alert users about satellite RFI in advance. \\
\hline
\textbf{Digital} & Cyclic Spectroscopy & Separates periodic astronomical signals from RFI. & GBT uses this technique to enhance pulsar studies and manage satellite RFI. \\
\hline
\textbf{Digital} & Sample Frequency Offset Sampling & Mitigates strong out-of-band RFI by decorrelating it during processing. & Applied in SKA Mid to reduce RFI artifacts from GSM, GNSS, and other transmitters. \\
\hline
\textbf{System} & Effelsberg Direct Digitization (EDD) & Fully digital receiver for flexible processing. & Supports multiple observation modes while testing SK-based RFI flagging. \\
\hline
\end{longtable}


%A summary of real-time techniques applicable to different types and sources of RFI including the location of excision in the receiver chain is provided in Table ~\ref{real-time-tech}. \\




%\begin{table}
 % \begin{center}
 % \caption{Summary of real-time techniques and mitigation location based on the type and source of RFI}
  %\label{real-time-tech}
% {\scriptsize
  %\begin{tabular}%{|l|c|c|c|}\hline 
%{\bf RFI Type} & {\bf Typical Source} & {\bf Mitigation At} & {\bf Technique} \\ 
%    \hline
%Strong, continuous,
% & Terrestrial Transmitters & Frontend  & Analog filters\\
 % Narrowband RFI  & (FM stations etc.) &  & (notch/band-reject) \\ \hline
 % Intermittent Narrowband & Terrestrial transmitters & Backend & Statistical RFI \\
%  RFI &  & (post-correlation,
 %& mitigation \\ 
%&  & frequency domain) & and excision
%Techniques \\ \hline
%Broadband RFI & Powerline RFI, & Backend & Statistical RFI \\
%&  sparking from & (pre-correlation, & mitigation \\ 
%&  vehicles & time domain) & and excision Techniques \\ \hline

%Spatially Confined & Extra-terrestrial & Antenna + Digital & Adaptive cancellation, \\
%RFI & (e.g. satellites) & Backend & spatial filtering, \\
%&  &  &  nulling \\ \hline

%silicon & $~0.1-10$ & enhanced $^{13}$C, $^{14}$N, $^{22}$Ne, s-process elem. & AGB stars & $> 90$~\% \\
%carbide & $~30$ & low $^{12}$C/$^{13}$C, often enh.\ $^{15}$N & J-type C-stars (?) & $< 5$~\% \\
% & & enhanced $^{12}$C, $^{15}$N, $^{28}$Si; extinct $^{26}$Al, $^{44}$Ti & Supernovae & 1~\% \\
% & & low $^{12}$C/$^{13}$C, low $^{14}$N/$^{15}$N & novae &  $0.1$~\% \\ \hline
%graphite & $~0.1-10$ & enh.\ $^{12}$C, $^{15}$N, $^{28}$Si; extinct $^{26}$Al, $^{41}$Ca, $^{44}$Ti &
% SN (WR?) & $< 80$~\% \\ 
% & $~10$ & s-process elements & AGB stars & $> 10$~\% \\
% & & low $^{12}$C/$^{13}$C  & J-type C-stars (?) & $< 10$~\% \\
% & & low $^{12}$C/$^{13}$C; Ne-E(L) & novae & 2~\% \\ \hline
% corundum/ & $~0.1-5$ & enhanced $^{17}$O, moderately depl. $^{18}$O & RGB / AGB & $> 70$~\% \\
% spinel/ & $~50$ & enhanced $^{17}$O, strongly depl. $^{18}$O & AGB stars & 20~\% \\
% hibonite & & enhanced $^{16}$O & supernovae & 1~\% \\ \hline
% silicates & $~0.1-1$ & similar to oxides above & & \\
 %&  $~140$ & & & \\ \hline
 %silicon & $~1$ & enhanced $^{12}$C, $^{15}$N, $^{28}$Si; extinct $^{26}$Al & supernovae & 100~\% \\
 %nitride & $~ 0.002$ & & & \\ \hline
%  \end{tabular}
%  }
% \end{center}
%\vspace{1mm}
% \scriptsize{
%{\it Notes:}\\
 % $^1$For the abund.\ (in wt.\ ppm) the reported maximum values from different meteorites are given. \\
%  $^2$Note uncertainty about actual fraction of diamonds that are pre-solar and for fraction of graphite attributed to SN and AGB stars (see discussion in text).}
%\end{table}